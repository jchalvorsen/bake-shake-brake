\documentclass[screen]{beamer}
\usepackage[T1]{fontenc}
\usepackage[latin1]{inputenc}



% Bruk NTNU-temaet for beamer (her i bokmålvariant), alternativer er
% ntnunynorsk og ntnuenglish.
\usetheme{ntnuenglish}
 
% Angi tittelen, vi gir også en kortere variant som brukes nederst på
% hver slide:
\title[The Finite Element Method]%
{Bake, Shake or Break: \\Breaking a wooden bridge}

% Denne kan du også bruke hvis det passer seg:
%\subtitle{Valgfri undertittel}

% Angir foredragsholder, også en (valgfri) kortversjon i
% hakeparanteser først som kommer nederst på hver slide:
\author[Halvorsen \& Hasund]{Halvorsen \& Hasund}

% Institusjon. Bruk gjerne disse slik det passer best med det du vil
% ha.  Valgfri kortversjon her også
\institute[NTNU]{The Finite Element Method project}

% Datoen blir også trykket på forsida. 
\date{16. November 2014}
%\date{} % Bruk denne hvis du ikke vil ha noe dato på forsida.

% Fra her av begynner selve dokumentet
\begin{document}
\setbeamertemplate{itemize items}[circle]

% Siden NTNU-malen har en annen bakgrunn på forsida, må dette gjøres
% i en egen kommando, ikke på vanlig beamer-måte:
\ntnutitlepage

% Her begynner første slide/frame, (nummer to etter forsida). 


\begin{frame}
\begin{columns}
    \begin{column}{\linewidth}
      \begin{block}{Modelling equations}
         \begin{align*}
         \frac{dc}{dt} &= \kappa \nabla^2 c \\
           \frac{dc}{dt} &= -k_1 c P^R + k_2 (1-P^R) \\
           \frac{dP^R}{dt} &=  -k_1 c P^R + k_2 (1-P^R)
         \end{align*}
      \end{block}
    \end{column}
  \end{columns}
\end{frame}



\end{document}