\section{Stress}

The code has so far only calculated the displacement in the geometry. We will now look at the stress $\bm{\sigma}$ which measures how much forces per area are acting on a particular spatial point. 

To calculate the stress in 3 dimensions, we used the stress matrices from \cite{stressMatrix}.


$\bm{C}$ gives us the relation between shear strain $\bm{\epsilon}$ and shear stress $\bm{\sigma}$.
\begin{align*}
\bm{\sigma} &= \bm{C} \, \bm{\epsilon}
\end{align*}


\begin{align*}
\bm{C} = \frac{E}{(1+\nu)(1-2\nu)}
\begin{bmatrix}
(1-\nu) & \nu & \nu & 0 & 0 & 0 \\
\nu & (1-\nu) & 0 & 0 & 0 & 0 \\
\nu & \nu & (1-\nu) & 0 & 0 & 0 \\
0 & 0 & 0 & \frac{1-2\nu}{2} & 0 & 0 \\
0 & 0 & 0 & 0 & \frac{1-2\nu}{2} & 0 \\
0 & 0 & 0 & 0 & 0 & \frac{1-2\nu}{2}
\end{bmatrix}
\end{align*}




\begin{align*}
\bm{\epsilon} = \bm{B} \bm{u}^e
\end{align*}

where $\bm{u}^e$ is the displacement field, $\bm{B}$ is the strain-displacement matrix

\begin{align*}
&\bm{B} = 
\begin{bmatrix} \\
\bar{\bm{\partial}} \bm{\phi_1} & \bar{\bm{\partial}} \bm{\phi_2} & \bar{\bm{\partial}} \bm{\phi_3} & \bar{\bm{\partial}} \bm{\phi_4} \\[1em]
\end{bmatrix}, \,&
\bar{\bm{\partial}} = 
\begin{bmatrix}
\frac{\partial}{\partial x} & 0 & 0 \\[0.3em]
0 & \frac{\partial}{\partial y} & 0 \\[0.3em]
0 & 0 & \frac{\partial}{\partial z} \\[0.3em]
\frac{\partial}{\partial y} & \frac{\partial}{\partial x} & 0 \\[0.3em]
\frac{\partial}{\partial z} & 0 & \frac{\partial}{\partial x}\\[0.3em]
0 & \frac{\partial}{\partial z} & \frac{\partial}{\partial y} \\
\end{bmatrix}
\end{align*}

and $\bm{\phi_i}$ is the shape function for the $i$-th node in the current element $\bm{u}^e$.

The resulting system for the stress is 
\begin{align*}
\bm{\sigma} = \bm{C} \bm{B} \bm{u}^e
\end{align*}

where 