\section{Bridge}

\begin{figure}
\center
\includegraphics[trim=0cm 5cm 7cm 7cm, clip=true, width=0.9\textwidth]{pic_bridge}
\caption{A picture of the bridge we built in minecraft.}
\label{fig:picBridge}
\end{figure}

For this section we ventured ahead and built ourself a bridge in minecraft as shown in figure \ref{fig:picBridge}. Importing the bridge into matlab were, apart from some tedious work, quite straight forward.

\subsection{Short overview of algorithm}
Since we wanted to have some kind of time-dependency on our chosen problem, we incorporated a car driving along the bridge. This required us to manually link together the meshes for the bridge and car, and is done in the \textit{mergeBridgeCar.m} matlab function. The supplied \textit{hex2tetr.m} function is used to split our minecraft cubes into fitting tetrahedrons. We find the boundary points at $z = 0$, and pass the whole thing into \textit{FEM.m} which solves the linear elasticity problem using the finite element method further specified in REF. The function \textit{stressRecovery} recovers the nodal Von Mises stresses by averaging the element  neighbor stresses for each node. The stress is found by REF.


\subsection{Materials chosen}
\begin{table}
\center
\caption{Material constants for oak wood used in the bridge.} 
\begin{tabular}{cc}
Material constant & Value \\ 
\hline 
Young's modulus & 11 GPa \\ 
Density & 650 kg/m$^3$ \\ 
Compressive yield strength & 46 MPa \\ 
Poissons ratio & 0.3 \\ 
\label{tab:oak}
\end{tabular}
\end{table}


For this problem we chose the material of the bridge to be oak wood. We defined the dimension of one minecraft block to be a cubic meter. The material constants we needed for wood were the averaged values we found at \cite{oak} and the value for the poissons ratio was conveniently set to 0.3. All the material constants are presented in table \ref{tab:oak}. We really wanted to see what happens when a heavy car drives over the bridge we defined the car to be extremely heavy.

