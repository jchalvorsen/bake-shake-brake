\section{Presentation of the 2D system}

First, let us present some basic results in 2D. Using the finite element method we have made a 2D solver for the linear elasticity problem on a domain $\Omega$. The domain is divided into triangles, called elements, each with 3 nodes. Each node represents an entry in the displacement vector

\begin{equation}
\bm{u}(x,y) = 
\begin{bmatrix}
u_x \\
u_y
\end{bmatrix},
\end{equation}

which is a measure of how much each spatial point has moved. The strain $\bm{\epsilon}$ measures how much each spatial point has deformed and the stress $\bm{\sigma}$ measures how much forces per area are acting on a spatial point. The relations between the displacement, strain and stress are listed in \cite{note2}, and we will go thoroughly through this for the 3D case. Before doing that, let us show that our 2D solver is correct by looking at the error when increasing the grid size.



