\section{Presentation of 3D model}
Modifying the solver to 3 dimensions, we get the relation 

\begin{equation}
\label{stress-strain}
\bm{\sigma} = \bm{C} \, \bm{\epsilon},
\end{equation}
where
\begin{align*}
\bm{\sigma} =
\begin{bmatrix}
\sigma_{xx} \\
\sigma_{yy} \\
\sigma_{zz} \\
\sigma_{xy} \\
\sigma_{yz} \\
\sigma_{xz}
\end{bmatrix}, \,
\bm{\epsilon} = 
\begin{bmatrix}
\epsilon_{xx} \\
\epsilon_{yy} \\
\epsilon_{zz} \\
\epsilon_{xy} \\
\epsilon_{yz} \\
\epsilon_{xz} \\
\end{bmatrix}
\end{align*}
and the stress-strain matrix
\begin{align*}
\bm{C} = \frac{E}{(1+\nu)(1-2\nu)}
\begin{bmatrix}
(1-\nu) & \nu & \nu & 0 & 0 & 0 \\
\nu & (1-\nu) & 0 & 0 & 0 & 0 \\
\nu & \nu & (1-\nu) & 0 & 0 & 0 \\
0 & 0 & 0 & \frac{1-2\nu}{2} & 0 & 0 \\
0 & 0 & 0 & 0 & \frac{1-2\nu}{2} & 0 \\
0 & 0 & 0 & 0 & 0 & \frac{1-2\nu}{2}
\end{bmatrix}.
\end{align*}
$E$ is the Young's modulus which is a measure of stiffness, and $\nu$ is the Poisson's ratio describing the ratio between the compression and expansion of a material. Further the relation between strain and displacement over one element is
\begin{align}
\label{strain-displacement}
\bm{\epsilon} = \bm{B} \bm{u}^e
\end{align}
where $\bm{u}^e$ is the displacement field over one element with four nodes, each with three spatial displacement directions and $\bm{B}$ is the strain-displacement matrix.
\begin{align*}
\bm{u}^e = 
\begin{bmatrix}
u_{1x} \\
u_{1y} \\
u_{1z} \\
\vdots \\
u_{4z}
\end{bmatrix}, \,
&\bm{B} = 
\begin{bmatrix} \\
\bar{\bm{\partial}} \bm{\phi_1} & \bar{\bm{\partial}} \bm{\phi_2} & \bar{\bm{\partial}} \bm{\phi_3} & \bar{\bm{\partial}} \bm{\phi_4} \\[1em]
\end{bmatrix} \textrm{ and } 
\bar{\bm{\partial}} = 
\begin{bmatrix}
\frac{\partial}{\partial x} & 0 & 0 \\[0.3em]
0 & \frac{\partial}{\partial y} & 0 \\[0.3em]
0 & 0 & \frac{\partial}{\partial z} \\[0.3em]
\frac{\partial}{\partial y} & \frac{\partial}{\partial x} & 0 \\[0.3em]
\frac{\partial}{\partial z} & 0 & \frac{\partial}{\partial x}\\[0.3em]
0 & \frac{\partial}{\partial z} & \frac{\partial}{\partial y} \\
\end{bmatrix}
\end{align*}
with $\bm{\phi_i}$ being the $i^{th}$ shape functions on the current element. The shape functions are in our case linear, and $\bm{\phi_i}$ must be 1 at the $i^{th}$ node and 0 at the three other nodes. Combining \eqref{stress-strain} and \eqref{strain-displacement}, we end up with the relation between displacement and stress,
\begin{align}
\label{stress-displacement}
\bm{\sigma} = \bm{C} \bm{B} \bm{u}^e.
\end{align}











